\documentclass[MTRX3700report.tex]{subfiles}
% Lydia

\begin{document}

\subsection{Module Requirements: Menu Navigation}
The operational scenarios considered place certain requirements on the something system, and on the modules that comprise it.
\subsubsection{Functional Requirements}
This section describes the functional requirements of Module X – those requirements that must be met if the module (and system) is to function correctly.  

\paragraph{Inputs}
Describe each external input, including signal encoding and timing, message encoding and timing, protocols, file formats, protection against input errors, etc, as relevant.
\paragraph{Process}
Describe the internal signal transformations and/or computer processing functionality required within the module, required performance limits, and error tolerances as appropriate.
\paragraph{Outputs}
Describe outputs that must be produced for the module to function correctly, including timing, frequency, protocols, etc as relevant.
\paragraph{Timing}
Any required timing or latency specifications that must be met.
\paragraph{Failure Modes}
Required functionality (if any) in the event of failure of various nominated components.

\subsubsection{Non-Function (Quality of Service) Requirements}
Non-functional requirements do not need to be met for the device to have basic function, but are required to provide specific levels of performance or engineering quality.
\paragraph{Performance}
Requirements such as computational loop time, accuracy, etc.
\paragraph{Interfaces}
Requirements such as computational loop time, accuracy, etc.
\paragraph{Design Constraints}
Practical or commercial considerations, such as programming languages, processor or other hardware, etc.

\subsection{Conceptual Design: Software Module Menu Navigation}
Now, for each module, give the outline of how it will work. In this section it is appropriate to present \\
•	The rationale for the design decisions that were made – why things were designed the way they were\\
•	block diagrams,\\
•	mathematical models and algorithms,\\
•	data flow diagrams\\
•	state-transition diagrams\\
•	listings of input and output formats\\
•	listings of message and data formats\\
•	responses to identifiable error conditions\\
•	responses to identifiable failure conditions\\
as appropriate for each module.

\subsubsection{Assumptions Made}
State any assumptions made.
\subsubsection{Constraints on Module X Performance}
State any constraints that may prevent the design from satisfying its requirements.




\end{document}