%\documentclass[MTRX3700report.tex]{subfiles}
\documentclass[11pt,a4paper]{article}
% Dpak
% HARDWARE DESIGN
\usepackage{color}
\usepackage{floatrow}
\usepackage{graphics}
\usepackage{graphicx}
% Table float box with bottom caption, box width adjusted to content
\newfloatcommand{capbtabbox}{table}[][\FBwidth]

\begin{document}
  \textcolor{red}
  {
  \textit{Give a detailed description of the design of hardware. The description should include mechanical drawings, location diagrams, electrical circuit schematics, circuit simulation or test results, PCB overlays, wiring diagrams, connector pinout lists, pneumatic/hydraulic circuit diagrams}
  }
  \\The project includes a wide array of hardware types, each with specific requirements to be interfaced with in software and with other the other hardware components. This section will seek to outline the requirements of the individual components used, as well as the scheme with which they were interfaced with each other.

  \subsection{Scope of the Jousting Robot's System Hardware}
    The jousting robot, being composed of 2 interacting systems, namely the Commander and Mobile Robot, has a hardware requirement that was also split into two.
    The Commander hardware and Mobile Robot hardware will be here discussed separately to highlight the differences.
  \subsection{Hardware shared between Commander and Mobile Robot}
    The Commander and Mobile robot function in a master/slave relatoionship and so required a common denominator of components for communication between them. These common components included the PIC18f4520 MNML microprocessor, and the Xbee communication hardware.
    \subsubsection{PIC18f4520 Microcontroller}
    The PIC18f4520 (shown in Figure~\ref{fig:PIC1}) is a RISC microcontroller with harvard architecture and is well suited to embedded projects. The MNML board used provides access to the MPLABx development environment through the RJ45 connector via the ICD (in circuit debugger). It is powered by 5V and can supply a maximum current of ~250mA, with each I/O pin being able to supply up to 25mA. These characteristics were enough for our application to the robotic system built.

    \begin{figure}[h]
        \includegraphics[width = 10cm]{pic.png}
        \caption{The MNML PIC18f4520 microcontroller}
        \label{fig:PIC1}
    \end{figure}
    The PIC features used in this project were:
    \begin{itemize}
      \item 3 external interrupt sources
      \item 4 timer modules
      \item 2 CCP (capture/compare/PWM) modules
      \item Addressable USART module compatible with RS232 standard
      \item 10 bit Analog to Digital converter module
    \end{itemize}

    \subsubsection{Xbee with regulator board}
      The Xbee hardware (shown in Figure~\ref{fig:xbee}) is a popular hobby radio communication device that came preconfigured to send and receive transmissions in pairs. That is, a source and destination address identifier was preset so each Xbee could only talk to the other one in the pair, to prevent miscommunications. The range of the device is 10m indoors and up to 30m outdoors. It communicates at a frequency of 2.4GHz and meets the IEEE 802.15.4 standards. The Xbee hardware came with a breakout board that doubled to provide power regulation features.
      \begin{figure}[h]
          \includegraphics[width=7cm]{xbee.png}
          \caption{The Xbee radio communication hardware}
          \label{fig:xbee}
      \end{figure}


  \subsection{Mobile Robot Hardware Design}
    \subsubsection{Power Supply}
      The Mobile Robot was powered by a Zippy Flightmax 4.2 Ahr Lithium Iron Phosphate (LiFePO4) battery. It consists of 4 series connected cells together producting a typical voltage of 12.8V. The battery is rated for a continuous discharge rate of 126A and a charge rate of 8.4A. The battery is connected to a power supply circuit that has several functions:
      \begin{enumerate}
        \item It provides a connection point for the 12.8V battery.
        \item It provides a connection point for external power supplies if the battery is removed.
        \item It provides 5V DC power at 1.5A to other hardware on the Mobile Robot.
        \item It provides a switch for turning power to the Mobile Robot off or on.
        \item It carries the Pololu 707 Dual VNH3SP30 Motor Driver board and provides power to it.
        \item It provides connection points for motor driver signals, motor leads and encoder signals.
      \end{enumerate}
    \subsubsection{Computer Design}
    Description of computer hardware, including all interface circuitry to sensors, actuators, and I/O hardware.
    \subsubsection{Sensor Hardware}
      \begin{enumerate}
        \item IR sensors:{ 3 Sharp GP2Y0A41SK0F Infrared Sensors were used to measure distance, with an effective range of 4-40cm, which corresponds to an output voltage of 0.4 to 3.1V. Exceeding 40cm, IR sensor output is saturated. There are 3 connections for the IR sensor,Vcc(nominally 5V), Gnd, and an analog output.}
        \item encoder:{ The rear motor shaft is fitted with a hall effect encoder which provides 2 square waves as outputs, one 90 degrees out of phase with the other. By measuring which signal is leading we can identify the direction of the wheel, and the frequency of the pulses tells us the speed. The sensor has 64 counts per revolution resolution. The Encoder was connected to a LM2907 frequency to voltage converter IC. It converted the pulses recieved into an analog voltage that the PIC could read via the ADC.
        \begin{figure}
          \centering
          \begin{minipage}{0.45\textwidth}
              \centering
              \includegraphics[width = 5cm]{encoder.png}
          \end{minipage}\hfill
          \begin{minipage}{0.45\textwidth}
              \centering
              \begin{tabular}[b]{|l|l|}
                \hline \textbf{Lead colour} & \textbf{Function}\\
                \hline Blue   & +5 V Supply\\
                \hline Green  & +0 V Supply (GND)\\
                \hline Yellow & A quadrature output signal\\
                \hline White  & B quadrature output signal\\
                \hline
              \end{tabular}
          \end{minipage}
          \caption{Encoder mounted on motor shaft, and wire functions}
        \end{figure}
        }
        \item Xbee (functions more like a sensor in robot)
      \end{enumerate}
    \subsubsection{Actuator Hardware}
      The primary actuators on the Mobile Robot were the two motors connected to the wheels. The two motors could be controlled independently allowing for turning functionality. The motor output shafts were connected to multistage planetary gearboxes having reduction ratios of 131.25:1.
      Table \ref{gearmotor} lists the acronyms and abbreviations used in this document.
      \begin{figure}[h]
        \begin{minipage}[b]{0.45\linewidth}
          \centering
          \includegraphics[width=5cm]{gearmotor.png}
        \end{minipage}\hfill
        \begin{minipage}[b]{0.45\linewidth}
          \begin{tabular}[b]{|l|l|}
            \hline \textbf{Parameter} & \textbf{Value} \\
            \hline Part number & Pololu 1447\\
            \hline No-load speed @ 12 V & 80 RPM\\
            \hline No-load current @ 12 V & 300 mA\\
            \hline Stall current @ 12 V & 5 A\\
            \hline Stall torque @ 12 V & 1.765 Nm\\
            \hline Positive terminal lead & Red\\
            \hline Negative terminal lead & Black\\
            \hline
          \end{tabular}
        \end{minipage}
        \caption{DC gearmotor and specifications}
        \label{gearmotor}
      \end{figure}
      The output shaft from the gearbox was connected to the 90mm wheel via 2 M3 screws that bore down on the shaft flat face.
    \subsubsection{Operator Input Hardware}
      The input to the Mobile Robot comes from the Commander via radio link, and as such there is no operator input hardware other than the power switch which is supplied by the power board circuit.
    \subsubsection{Operator Output Hardware}
      The operator recieves output from the Mobile Robot in the form of motion, and so the motors are an output. All other information about the state of the system is relayed via radio link to the Commander. The Mobile Robot also has a battery voltage indicator which allows for quick diagnosis of the charge level of the battery.
    \subsubsection{Hardware Quality Assurance}
      The power requirements of the Mobile Robot are regulated and met by the circuitry inside the power board. It also features a lower battery voltage cutoff of 11V to prevent the battery of excessive discharge. The IR sensor is sensitive and to stop unstable values from being read, the IR sensors were mounted onto the Mobile Robot at fixed locations. The PIC also contains brownout protection, and upon reset the code was written to keep the Mobile Robot stationary until instructed otherwise.\\

    Describe any measures that were taken to control (improve) hardware quality and reliability – Heartbeats, brownout conditioning/resets, reset conditions, testing and validation, etc.

  \subsection{Commander Hardware Design}
    The design of the Commander changed as the project went on, and the final product that was arrived at was significantly different due to design choices between different sections of the project. The evolution of the design is shown in Figure~\ref{fig:commanderDesign}.

      \begin{figure}
        \caption{\textbf{Initial Commander}}
        \includegraphics[width = 15cm]{initialCommander.png}
      \end{figure}
      \vspace{25pt}
      \begin{figure}
        \caption{\textbf{Intermediate Design}}
        \centering
        \includegraphics[width=10cm]{draftCommander.png}
      \end{figure}
      \vspace{25pt}
      \\
      \begin{figure}
        \centering
        \caption{\textbf{Final Commander Design}}
        \includegraphics[width=10cm]{finalCommander.png}
        \label{fig:commanderDesign}
      \end{figure}

    \end{figure*}
    \subsubsection{Power Supply}
      Unlike the Mobile Robot, the power requirements of the Commander could be fulfilled from a single 9v battery plugged into the PIC, and all power for peripheral components supplied by the PIC. Due to the many components connected to the power lines, a power rerouting circuit needed to be constructed. This circuit is shown in Figure~\ref{fig:commanderVero}.
      \begin{figure}
        \includegraphics[width = 15cm]{commanderVero.png}
        \caption{The veroboard circuit built for the commander}
        \label{fig:commanderVero}
      \end{figure}
    \subsubsection{Computer Design}
      Description of computer hardware, including all interface circuitry to sensors, actuators, and I/O hardware.
    \subsubsection{Sensor Hardware}
    The main sensor on the Commander is the Joystick, which is 2 dual axis potentiometers connected perpendicular to each other. A breakout board was used to enable easier access to the pins.
    \begin{figure}[h]
      \includegraphics[width=8cm]{joystick.png}
      \caption{The joystick used on the Commander}
      \label{fig:joystick}
    \end{figure}
    \subsubsection{Actuator Hardware}
      There was no actuating hardware present on the Commander, however, the Xbee hardware served as a means of communication between Commander and Mobile Robot.
    \subsubsection{Operator Input Hardware}
      Input hardware consisted of a 2 axis joystick, which is 2 potentiometers at mounted right angles to each other. This joystick also contains a button that can be used by depressing the joystick. A second, separate button was used on the commander for starting and stopping the mobile robot. The commander also included a master power switch. The hardware is shown in Figure~\ref{fig:joystick}
      \pagebreak
    \subsubsection{Operator Output Hardware}
      \paragraph{LCD}
      The main output hardware consisted of an LCD screen which displayed the state of the system, as well as basic information about the Mobile Robot to to the operator. In the interest of saving pins, the system uses the LCD in 4-bit mode.
      The LCD hardware consists of 14-inputs given as follows:
      \begin{table}[h]
        \begin{tabular}{|p{6cm}|p{5cm}|}
          \hline \textbf{LCD pin} & \textbf{Connection}\\
        	\hline Ground & Gnd\\
        	\hline Vcc & 5V\\
        	\hline Contrast & Output of 10KOhm potentiometer\\
        	\hline RS Register Select - 0 = Instruction; 1 = data read/write & PORTDbits.RD2\\
        	\hline RW Read/Write  & Gnd)\\
        	\hline Enable & PORTDbits.RD3\\
        	\hline Data bits 0-3 & Gnd in 4-bit mode\\
        	\hline Data bits 4-7 & PORTDbits.RD4-7\\
          \hline
        \end{tabular}
      \end{table}
      \begin{figure}[h!]
      	\includegraphics[scale=0.6]{LCD.jpg}
      	\centering
      	\caption{LCD inputs}
      \end{figure}
      3 LEDs were also in the original design, a power indicator, a radio link integrity indicator, and an LED to indicate the Mobile Robot was in Autonomous mode, however these LED indicators were not implemented in the final product.
      \pagebreak
    \subsubsection{Hardware Quality Assurance}
      The main hardware quality assurance policy was implemented on the PIC, with a brownout detector enabled by default on the microcontroller.\\

  \subsection{Hardware Validation}

    \subsubsection{Commander}
    \begin{itemize}
      \item The first thing implemented was the serial port, so that the readings from the other peripheral hardware could be validated, both while connected to a local PC or wirelessly to the other PIC.
      \item The joystick axis values were validated first with an oscilloscope, to check whether the magnitudes were correct and in the right direction. Next, the values were sent via serial to a local PIC running Teraterm. The next step was to send the joystick values to the Mobile Robot and check if they were still acurately recieved.
      \item The LCD was systematically tested by sending progressively more complicated strings to it until the final implementation was reached.
      \item The Xbee came preconfigured and so the only validation performed was to check if signals were being transmitted and recieved accurately, which they were.
      \item Both the joystick button and the other button were connected to pull up resistors to ensure no undefined behaviour was interpreted by the PIC.
    \end{itemize}
    \subsubsection{Mobile Robot}
      \begin{itemize}
        \item The IR LED was tested firstly with a scope to check that the voltage was in the correct magnitude and range. Next, the IR signal was passed to the ADC on the PIC and displayed via serial to a local PC running Teraterm to ensure that the ADC was reading the values correctly.
        \item The Pololu H bridge circuit was validated by probing the Vin and Gnd pins with a multimeter, as well as probing the OUT xA and B pins with an oscilloscope to check that motoros were getting appropriate signals when the PIC tried to communicate with them.
        \item The motor was tested by checking if the connections to the shaft were secure. This was a regular problem and is part of the maintenance routine.
        \item The encoders were tested using the frequency to voltage converter IC.
        \item The power board was initially tested with lab power supplies to make sure that correct voltages were outputted when the input was appropriate. The various output voltages being supplied to the motor driver, the PIC, and the IR leds as well as auxillary power outputs were also probed with a multimeter to ensure correct behaviour.

      \end{itemize}




\end{document}
