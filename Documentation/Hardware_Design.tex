\documentclass[MTRX3700report.tex]{subfiles}
% Dpak
% HARDWARE DESIGN
\begin{document}

  \textit{Give a detailed description of the design of hardware. The description should include mechanical drawings, location diagrams, electrical circuit schematics, circuit simulation or test results, PCB overlays, wiring diagrams, connector pinout lists, pneumatic/hydraulic circuit diagrams}
  The project includes a wide array of hardware types, each with specific requirements to be interfaced with in software and with other the other hardware components. This section will seek to outline the requirements of the individual components used, as well as the scheme with which they were interfaced with each other.

  \subsection{Scope of the Jousting Robot's System Hardware}
  The jousting robot, being composed of 2 interacting systems, namely the Commander and Mobile Robot, has a hardware requirement that was also split into two.
  The Commander hardware and Mobile Robot hardware will be here discussed separately to highlight the differences.
  \subsection{Commander Hardware Design}
    \subsubsection{Power Supply}
    Power supply method and rating, fusing, distribution, grounding and protective earth as appropriate.
    \subsubsection{Computer Design}
    Description of computer hardware, including all interface circuitry to sensors, actuators, and I/O hardware.
    \subsubsection{Sensor Hardware}
    \subsubsection{Actuator Hardware}
    \subsubsection{Operator Input Hardware}
    \subsubsection{Operator Output Hardware}
    \subsubsection{Hardware Quality Assurance}
    Describe any measures that were taken to control (improve) hardware quality and reliability – Heartbeats, brownout conditioning/resets, reset conditions, testing and validation, etc.

  \subsection{Mobile Robot Hardware Design}
    \subsubsection{Power Supply}
    Power supply method and rating, fusing, distribution, grounding and protective earth as appropriate.
    \subsubsection{Computer Design}
    Description of computer hardware, including all interface circuitry to sensors, actuators, and I/O hardware.
    \subsubsection{Sensor Hardware}
    \subsubsection{Actuator Hardware}
    \subsubsection{Operator Input Hardware}
    \subsubsection{Operator Output Hardware}
    \subsubsection{Hardware Quality Assurance}
    Describe any measures that were taken to control (improve) hardware quality and reliability – Heartbeats, brownout conditioning/resets, reset conditions, testing and validation, etc.


  \subsection{Hardware Validation}
    \subsubsection{Commander}
    \subsubsection{Mobile Robot}
  Details of any systematic testing to ensure that the hardware actually functions as intended.

  \subsection{Hardware Calibration Procedures}
  Procedures for calibration required in the factory, or in the field.
    \subsubsection{Commander}
    \subsubsection{Mobile Robot}

  \subsection{Hardware Maintenance and Adjustment}
  Routine adjustment and maintenance procedures.
    \subsubsection{Commander}
    \subsubsection{Mobile Robot}


\end{document}
