\documentclass[MTRX3700report.tex]{subfiles}


\begin{document}
\setcounter{subsection}{1} % leave here


%\section{System Performance}
\subsubsection{Performance Testing}
%Give the results of testing conducted to determine the characteristics and performance of the system - memory usage, loop time, system accuracy, repeatability, ease of use, etc.
In manual mode testing found that their was a delay between the joystick commands from the controller to the robot. The communications between the robot and the controller also occasionally broke down preventing the user from controlling the robot with the joystick. However the user could still turn on and off the robot motors through the run button.

In full auto mode testing was done using different turning values and speeds by trial and error to find suitable values. The robot operated as directed however there was a large delay when changing direction with the motors.
\subsubsection{State of the System as Delivered}
%A statement of your group’s opinion of the conformance of the system with the specification.
The system dis not meet all the specifications when delivered. The major downfall was that the communications between the robot and commander broke down with no obvious reason and that the robot continued running when the controller was turned off. There was also no PID or other method of control for the motors.
\subsubsection{Future Improvements}
%Present a prioritised list of improvements to be made in future releases, giving reasons for the improvement and priority rank.
\begin{enumerate}
	\item Fix control for motors
	\item Implement working handshake function so the robot stops when the controller is suddenly turned off
	\item Make use of front IR sensor to stop the robot before it hits an obstacle
	\item Add an additional IR sensor to the side to improve performance of Auto mode 
	\item Implement a working User assist mode which keeps the robot parallel to the tilt while the user drives it.
\end{enumerate}

\end{document}